%% LyX 2.0.6 created this file.  For more info, see http://www.lyx.org/.
%% Do not edit unless you really know what you are doing.
\documentclass[twoside]{sig-alternate}
\usepackage[T1]{fontenc}
\usepackage[latin9]{inputenc}
\usepackage{color}
\usepackage{verbatim}
\usepackage{prettyref}
\usepackage{float}
\usepackage{amsmath}
\usepackage{amssymb}
\usepackage[authoryear]{natbib}

\makeatletter

%%%%%%%%%%%%%%%%%%%%%%%%%%%%%% LyX specific LaTeX commands.
\floatstyle{ruled}
\newfloat{algorithm}{tbp}{loa}
\providecommand{\algorithmname}{Algorithm}
\floatname{algorithm}{\protect\algorithmname}

%%%%%%%%%%%%%%%%%%%%%%%%%%%%%% Textclass specific LaTeX commands.
\floatstyle{ruled}
\newfloat{algorithm}{tbp}{loa}
\floatname{algorithm}{Algorithm}
\usepackage{algorithmic}
\newcommand{\forbody}[1]{ #1 \ENDFOR }
\newcommand{\ifbody}[1]{ #1  \ENDIF}
\newcommand{\whilebody}[1]{ #1  \ENDWHILE}
\renewcommand{\algorithmicprint}{\textbf{draw}}
\renewcommand{\algorithmicrequire}{\textbf{Input:}}
\renewcommand{\algorithmicensure}{\textbf{Output:}}

 \newtheorem{thm}{Theorem}[section]
 \newtheorem{lem}[thm]{Lemma}
 \newtheorem{defn}[thm]{Definition}
 \newtheorem{exmp}[thm]{Example}

%%%%%%%%%%%%%%%%%%%%%%%%%%%%%% User specified LaTeX commands.

%%%%%%%%%%%%%%%%%%%%%%%%%%%%%% Textclass specific LaTeX commands.
\usepackage{algorithm}\usepackage{algorithmic}\floatstyle{ruled}
\newfloat{algorithm}{tbp}{loa}
\floatname{algorithm}{Algorithm}
\usepackage{algorithmic}%\newcommand{\forbody}[1]{ #1 \ENDFOR }
%\newcommand{\ifbody}[1]{ #1  \ENDIF}
%\newcommand{\Comment}[1]{ }
\newcommand{\Comment}[1]{ #1}
%\newcommand{\whilebody}[1]{ #1  \ENDWHILE}
\renewcommand{\algorithmicprint}{\textbf{draw}}
\renewcommand{\algorithmicrequire}{\textbf{Input:}}
\renewcommand{\algorithmicensure}{\textbf{Output:}}
\newcommand{\sO}[2]{{#1}^{#2+o(1)}}
\newcommand{\sOsO}[4]{{#1}^{#2+o(1)}\*{#3}^{#4+o(1)}}
\newcommand{\OsO}[4]{{#1}^{#2}\*{#3}^{#4+o(1)}}
\newcommand{\sOsconst}[3]{{#1}^{#2+o(1)}\*{#3}^{o(1)}}


%\newtheorem{thm}{Theorem}
\newtheorem{cor}{Corollary}
\newtheorem{prop}{Proposition}
%\newtheorem{defn}{Definition}
\newtheorem{Algo}{Algorithm}
%\newtheorem{example}{Example}
%\newtheorem{defn}[thm]{Definition}
\newtheorem{rem}{Remark}
%\newtheorem{lem}{Lemma}


\newenvironment{Alg}{\noindent  {\bf Algorithm} \hspace*{0.05cm}}{

}\newcommand{\Z}{\mathbb{Z}}
\newcommand{\Q}{\mathbb{Q}}
\newcommand{\N}{\mathbb{N}}
\newcommand{\K}{ {\rm K}}
\newcommand{\revCol}{ {\rm revCol}}
\newcommand{\bigO}{\mathcal{O}}
\newcommand{\tbigO}{\widetilde{\mathcal{O}}}
\newcommand{\cL}{\mathcal{L}}
\newcommand{\ocL}{\overline{\mathcal{L}}}
\newcommand{\tcL}{\widetilde{\mathcal{L}}}
\newcommand{\cA}{\mathcal{A}}
\newcommand{\oA}{\overline{A}}
\newcommand{\GL}{{\rm GL}\,}
\newcommand{\rank}{{\rm rank}\,}
\newcommand{\reverse}{{\rm Reverse}\,}
\newcommand{\cdeg}{{\rm cdeg}\,}
\newcommand{\rdeg}{{\rm rdeg}\,}
\newcommand{\diag}{{\rm diag}\,}
\newcommand{\val}{{\rm val}\,}
\newcommand{\order}{{\rm ord}\,}
\newcommand{\abs}[1]{\lvert#1\rvert}

\newcommand{\Return}{\textbf{Return}}
\newcommand{\While}{\textbf{While}}
\newcommand{\For}{\textbf{For}}
\newcommand{\If}{\textbf{If}}
\newcommand{\nv}{\textbf{MatrixPolynomialInverse}}
\newcommand{\then}{\textbf{then}}
\newcommand{\ddo}{\textbf{do}}
\newcommand{\edo}{\textbf{end do}}
\newcommand{\eif}{\textbf{end if}}
\newcommand{\cdim}{{\rm coldim}}
\newcommand{\tO}{O^{\sim}}

\def\StorjohannTransform{\qopname\relax n{StorjohannTransform}}
\def\TransformUnbalanced{\qopname\relax n{TransformUnbalanced}}
\def\rowDimension{\qopname\relax n{rowDimension}}
\def\columnDimension{\qopname\relax n{columnDimension}}
\DeclareMathOperator{\re}{rem}
\DeclareMathOperator{\coeff}{coeff}
\DeclareMathOperator{\lcoeff}{lcoeff}
\DeclareMathOperator{\inv}{inverse}
\DeclareMathOperator{\rev}{rev}
\DeclareMathOperator{\colRev}{colRev}
\DeclareMathOperator{\rowRev}{rowRev}
\DeclareMathOperator{\unimodularCompletion}{unimodularCompletion}
\DeclareMathOperator{\hermiteDiagonal}{hermiteDiagonal}
\DeclareMathOperator{\mnbr}{MinimalKernelBasisTranspose}
\DeclareMathOperator{\mnbrp}{MinimalKernelBasisWithRankProfile}
\DeclareMathOperator{\colBasis}{ColumnBasis}
\DeclareMathOperator{\rankProfile}{rankProfile}
\def\mab{\qopname\relax n{OrderBasis}}
\def\mnbrs{\qopname\relax n{MinimalNullspacerBasisRankSensitive}}
\def\mmab{\qopname\relax n{FastBasis}}
\def\umab{\qopname\relax n{UnbalancedFastOrderBasis}}
\def\mnb{\qopname\relax n{MinimalKernelBasis ~ }}
\newcommand{\bb}{\\}

%\input{macro}
% variables
\newcommand{\dg}{r}
\newcommand{\dz}{n}
\newcommand{\A}{\ensuremath{A}}
\newcommand{\I}{\ensuremath{I}}
\newcommand{\B}{\ensuremath{B}}
%\newcommand{\U}{\ensuremath{U}}
\newcommand{\V}{\ensuremath{V}}
\newcommand{\W}{\ensuremath{W}}
\newcommand{\Her}{\ensuremath{H}}
\newcommand{\Hi}{\ensuremath{H}}
\newcommand{\Sm}{\ensuremath{S}}
\newcommand{\Vn}{\ensuremath{V_{\mathfrak{n}}}}
\newcommand{\Vi}{\ensuremath{V_{\mathfrak{i}}}}
\newcommand{\Wu}{\ensuremath{W_{\mathfrak{u}}}}
\newcommand{\Wd}{\ensuremath{W_{\mathfrak{d}}}}
\newcommand{\An}{\ensuremath{A_{\mathfrak{n}}}}
\newcommand{\Ai}{\ensuremath{A_{\mathfrak{i}}}}
\newcommand{\Au}{\ensuremath{A_{\mathfrak{u}}}}
\newcommand{\Ad}{\ensuremath{A_{\mathfrak{n}}}}
\renewcommand{\Pi}{\ensuremath{P_{\mathfrak{i}}}}
\newcommand{\Pn}{\ensuremath{P_{\mathfrak{n}}}}
\newcommand{\Pu}{\ensuremath{P_{\mathfrak{u}}}}
\newcommand{\Pd}{\ensuremath{P_{\mathfrak{d}}}}

% Groups
\newcommand{\gva}{\ensuremath{\mathcal{G}}}







\def\qed{\hspace*{2mm} \hfill $\Box $\bigskip}
\newcommand{\ds}{\displaystyle}
\newrefformat{alg}{Algorithm \ref{#1}}

\def\doi#1{\gdef\@doi{#1}}\def\@doi{}
\global\copyrightetc{%Copyright \the\copyrtyr\ 
ACM \the\acmcopyr\ ...\$15.00}
\toappear{\the\boilerplate\par{\confname{\the\conf}} \the\confinfo\par\the\copyrightetc.\ifx\@doi\@empty\else\par\@doi.\fi}
\newfont{\mycrnotice}{ptmr8t at 7pt}
\newfont{\myconfname}{ptmri8t at 7pt}
\let\crnotice\mycrnotice%
\let\confname\myconfname%

\makeatother

\begin{document}


% --- Author Metadata here ---
% Permission Statement
\permission{Permission to make digital or hard copies of all or part of this work for personal or classroom use is granted without fee provided that copies are not made or distributed for profit or commercial advantage and that copies bear this notice and the full citation on the first page. Copyrights for components of this work owned by others than the author(s) must be honored. Abstracting with credit is permitted. To copy otherwise, or republish, to post on servers or to redistribute to lists, requires prior specific permission and/or a fee. Request permissions from Permissions@acm.org.}

% Conference Information
\conferenceinfo{ISSAC '14,}{July 21 - 25, 2014, Kobe, Japan\\ 
Copyright is held by the owner/author(s). Publication rights licensed to ACM.}
\crdata{978-1-4503-2501-1/14/07}
% DOI
\doi{http://dx.doi.org/10.1145/2608628.2608640}


\author{\alignauthor Wei Zhou and George Labahn\\
 \affaddr{Cheriton School of Computer Science}\\
 \affaddr{University of Waterloo}, \\
 \affaddr{Waterloo, Ontario, Canada}\\
 \email{\{w2zhou,glabahn\}@uwaterloo.ca} }


\date{{\normalsize{{{{\today \quad{}:: \timeofday}}}}}}


\title{Unimodular Completion of Polynomial Matrices}
\maketitle
\begin{abstract}
Given a rectangular matrix $\mathbf{F}\in\mathbb{K}[x]^{m\times n}$
with $m<n$ of univariate polynomials over a field $\mathbb{K}$ we
give an efficient algorithm for computing a unimodular completion
of $\mathbf{F}$. %That is, our algorithm finds a polynomial matrix $\mathbf{G}\in\mathbb{K}[x]^{(n-m)\times n}$
%such that the square matrix $\left[\begin{array}{c}
%\mathbf{F}\\
%\mathbf{G}
%\end{array}\right]$ is unimodular. 
Our algorithm is deterministic and computes such a completion, when
it exists, with cost $O^{\sim}\left(n^{\omega}s\right)$ field operations
from $\mathbb{K}$. Here $s$ is the average of the $m$ largest column
degrees of $\mathbf{F}$ and $\omega$ is the exponent on the cost
of matrix multiplication. Here $O^{\sim}$ is big-O but with log factors
removed. If a unimodular completion does not exist for $\mathbf{F}$,
our algorithm computes a unimodular completion for a right cofactor
of a column basis of $\mathbf{F}$, or equivalently, computes a completion
that preserves the generalized determinant. % defined as the product of the entries of the Smith normal form. 

\end{abstract}
\vspace{1mm}
 \textbf{Categories and Subject Descriptors:} I.1.2 {{[}{Symbolic
and Algebraic Manipulation}{]}}: {Algorithms}; F.2.2 {{[}{Analysis
of Algorithms and Problem Complexity}{]}}: {Nonnumerical Algorithms
and Problems}

\noindent \vspace{1mm}
 \textbf{General Terms:} Algorithms, Theory

\noindent \vspace{1mm}
 \textbf{Keywords:} Order Basis, Minimal Kernel Basis, Unimodular
Matrices, Unimodular Completion


\section{\label{sec:intro}Introduction}

Let $\mathbf{F}\in\mathbb{K}\left[x\right]^{m\times n}$ with $m<n$
be a full-rank rectangular matrix. We consider the unimodular completion
problem of finding a second rectangular matrix $\mathbf{G}\in\mathbb{K}\left[x\right]^{(n-m)\times n}$
such that $\left[\begin{array}{c}
\mathbf{F}\\
\mathbf{G}
\end{array}\right]$ is unimodular. Here a square matrix of univariate polynomials is
unimodular if its determinant is a nonzero constant (and so also has
an inverse consisting of polynomials). In fact, we consider the more
general %completion 
problem of finding $\mathbf{G}\in\mathbb{K}\left[x\right]^{(n-m)\times n}$
such that the $\mathbf{F}$ and $\left[\begin{array}{c}
\mathbf{F}\\
\mathbf{G}
\end{array}\right]$ have the same \emph{generalized determinant}. Here the generalized
determinant denotes the product of the nonzero diagonal elements of
its Smith form. The standard unimodular completion problem is then
the special case where the generalized determinant of $\mathbf{F}$
is $1$. %
\begin{comment}
We also wish to have a type of degree minimality for our completion
problem, that is, that the degree of our completion are consistent
with the degrees of our input matrix. 
\end{comment}
{} %By this we mean if $\mathbf{F}$ has column degrees bounded by a vector of nonnegative integers $\vec{s}$ then we also require that $\mathbf{G}^{T}$ has  $\left(-\vec{s}\right)$-minimal rows. 


In general unimodular completion is a useful basic operation in many
matrix computations \citep{newman1972}. For example, in the case
of a single row (i.e. $m=1$) unimodular completion plays a key role
in proofs for Hermite and Smith normal forms. In the case of rectangular
matrices of multivariate polynomials, the problem has more storied
history because of its role in solving Serre's conjecture that projective
modules over polynomial rings are free \citep{Lam1978}. Serre's conjecture
was proved independently by Quillen and Suslin in 1976 using unimodular
completion \citep{YoulaPickel}. Later on their nonconstructive methods
were replaced by constructive methods, for example that of Logar and
Sturmfels \citep{Sturmfels}. Unimodular completion also appears in
non commutative domains, specifically where their use includes the
reduction of mathematical systems to ones having fewer functions and
parameters \citep{CluzeauQuadrat}. Additional applications in multidimensional
systems theory, including a description of a Maple package which implements
the Quillen-Suslin theorem, can be found in \citep{Quadrat}.

Unimodular completions do not exist for all rectangular matrices.
For example, the matrix $\left[0,x\right]$ cannot be completed to
a $2\times2$ unimodular matrix. In the case where $\mathbf{F}$ is
a single row, a unimodular completion exists if and only if all the
entries in $\mathbf{F}$ are relatively prime \citep{newman1972}.
More generally we can show that a unimodular completion of $\mathbf{F}$
exists if and only if there exists a unimodular matrix $\mathbf{U}$
such that $\mathbf{F}\cdot\mathbf{U}=[I_{m},0].$ For simplicity and
without loss of generality, we focus our discussion on the situation
where the unimodular completion always exists. For an input matrix
that cannot be completed to a unimodular matrices, we can always factor
it as $\mathbf{F}=\mathbf{T}\mathbf{G}$ using the column basis algorithm
from \citep{zl2013}, something which can be done efficiently. Then
the right factor $\mathbf{G}$ can be unimodularly completed. However,
an even simpler way is to apply our algorithm directly on the input
matrix, as the algorithm works effectively on any input matrix to
compute a completion that preserves the generalized determinant.

The algorithm we present in this paper has a cost of $O^{\sim}\left(n^{\omega}s\right)$
field operations, where $s$ is the average of the $m$ largest column
degrees of $\mathbf{F}$. Our approach is to embed our matrix into
an {\em order basis} problem \citep{BeLa94}%
\begin{comment}
in a similar way to that used in the shifted minimal kernel basis
algorithm of \citep{za2012} 
\end{comment}
. %
\begin{comment}
In our case w 
\end{comment}
We take advantage of the fact that order bases are closely related
to unimodular matrices when one reverses the order of column coefficients.
More precisely, we first reverse coefficients of $\mathbf{F}$, then
compute a kernel basis $\mathbf{M}$ of this new object, and a left
order basis $\mathbf{Q}$ of $\mathbf{M}$. Reversing coefficients
in $\mathbf{Q}$ then gives a unimodular completion of $\mathbf{F}$.%
\begin{comment}
, one part essentially gives back $\mathbf{F}$, and the other one
%the missing piece 
$\mathbf{G}$ 
\end{comment}
{} A major challenge lies in determining the right shifts and orders
so that the pieces both fit together and give a low complexity.


\section{Preliminaries}

In this section we give the basic cost model, notations, and the basic
definitions and properties of {\em shifted degree}, {\em kernel
basis}, \emph{column basis, }and {\em order basis} which are needed
for our discussion and algorithm. %
\begin{comment}
In addition, we discuss the various ways one can reverse the coefficients
of matrix polynomials along with the properties of such reversals. 
\end{comment}



\subsection{Cost model}

Algorithms are analyzed by bounding the number of arithmetic operations
in the coefficient field $\mathbb{K}$ on an algebraic random access
machine. We will frequently use the fact that the cost of multiplying
two polynomial matrices with dimension $n$ and degree bounded by
$d$ is $O^{\sim}(n^{\omega}d)$ field operations from $\mathbb{K}$,
where $\omega$ is the exponent of matrix multiplication. We refer
to the book by \citep{vonzurgathen} for more details and references
about polynomial and matrix multiplication.


\subsection{Notations}

For convenience we adopt the following notations in this paper. 
\begin{description}
\item [{Comparing~Unordered~Lists}] For two lists $\vec{a}\in\mathbb{Z}^{n}$
and $\vec{b}\in\mathbb{Z}^{n}$, let $\bar{a}=\left[\bar{a}_{1},\dots,\bar{a}_{n}\right]\in\mathbb{Z}^{n}$
and $\bar{b}=\left[\bar{b}_{1},\dots,\bar{b}_{n}\right]\in\mathbb{Z}^{n}$
be the lists consists of the entries of $\vec{a}$ and $\vec{b}$
but sorted in increasing order. 
\[
\begin{cases}
\vec{a}\ge\vec{b} & \mbox{if }\bar{a}_{i}\ge\bar{b}_{i}\mbox{ for all }i\in\left[1,\dots,n\right]\\
\vec{a}\le\vec{b} & \mbox{if }\bar{a}_{i}\le\bar{b}_{i}\mbox{ for all }i\in\left[1,\dots,n\right]\\
\vec{a}>\vec{b} & \mbox{if }\vec{a}\ge\vec{b}\mbox{ but }\bar{a}_{j}\ne\bar{b}_{j}\\
\vec{a}<\vec{b} & \mbox{if }\vec{a}\le\vec{b}\mbox{ but }\bar{a}_{j}\ne\bar{b}_{j}
\end{cases}
\]
\begin{comment}
Summation~Notation For a list $\vec{a}=\left[a_{1},\dots,a_{n}\right]\in\mathbb{Z}^{n}$,
we write $\sum\vec{a}$ without index to denote the summation of all
entries in $\vec{a}$. 
\end{comment}
%\item [{}]~

\item [{Uniform~Shift~of~a~List}] For a list $\vec{a}=\left[a_{1},\dots,a_{n}\right]\in\mathbb{Z}^{n}$
and $c\in\mathbb{Z}$, we write $\vec{a}+c$ to denote $\vec{a}+\left[c,\dots,c\right]=\left[a_{1}+c,\dots,a_{n}+c\right]$,
with subtraction handled similarly. 
\item [{Compare~a~List~with~an~Integer}] For %a list 
$\vec{a}=\left[a_{1},\dots,a_{n}\right]\in\mathbb{Z}^{n}$ and $c\in\mathbb{Z}$,
we write $\vec{a}<c$ to denote $\vec{a}<\left[c,\dots,c\right]$,
and similarly for $>,\le,\ge,=$. 
\end{description}

\subsection{Shifted degrees}

Our methods depend extensively on the concept of {\em shifted}
degrees of polynomial matrices \citep{BLV:1999}. For a column vector
$\mathbf{p}=\left[p_{1},\dots,p_{n}\right]^{T}$ of univariate polynomials
over a field $\mathbb{K}$, its column degree, denoted by $\cdeg\mathbf{p}$,
is the maximum of the degrees of the entries of $\mathbf{p}$, that
is, 
\[
\cdeg\mathbf{p}=\max_{1\le i\le n}\deg p_{i}.
\]
The \emph{shifted column degree} generalizes this standard column
degree by taking the maximum after shifting the degrees by a given
integer vector that is known as a \emph{shift}. More specifically,
the shifted column degree of $\mathbf{p}$ with respect to a shift
$\vec{s}=\left[s_{1},\dots,s_{n}\right]\in\mathbb{Z}^{n}$, or the
\emph{$\vec{s}$-column degree} of $\mathbf{p}$ is 
\[
\cdeg_{\vec{s}}\mathbf{p}=\max_{1\le i\le n}[\deg p_{i}+s_{i}]=\cdeg(x^{\vec{s}}\cdot\mathbf{p}),
\]
where 
\[
x^{\vec{s}}=\diag\left(x^{s_{1}},x^{s_{2}},\dots,x^{s_{n}}\right).
\]
For a matrix $\mathbf{P}$, we use $\cdeg\mathbf{P}$ and $\cdeg_{\vec{s}}\mathbf{P}$
to denote respectively the list of its column degrees and the list
of its shifted $\vec{s}$-column degrees. When $\vec{s}=\left[0,\dots,0\right]$,
the shifted column degree specializes to the standard column degree.
The shifted row degree is defined in a similar way.

Shifted degrees have been used previously in polynomial matrix computations
and in generalizations of some matrix normal forms \citep{BLV:jsc06}.
The shifted column degree is equivalent to the notion of \emph{defect}
commonly used in the literature.

Along with shifted degrees we also make use of the notion of a polynomial
matrix being column reduced. A polynomial matrix $\mathbf{A}\in\mathbb{K}\left[x\right]^{m\times n}$
is column reduced if the leading column coefficient matrix, that is
the matrix 
\[
\lcoeff_{\vec{d}}\mathbf{A}=[\coeff(a_{ij},x,d_{j})]_{1\leq i\leq m,1\leq j\leq n},\mbox{ with }\vec{d}=\cdeg\mathbf{A},
\]
has full rank. A polynomial matrix $\mathbf{A}$ is $\vec{s}$-column
reduced if $x^{\vec{s}}\cdot\mathbf{A}$ is column reduced.

The usefulness of the shifted degrees can be seen from their applications
in polynomial matrix computation problems \citep{zhou:phd2012,ZL2012,za2012}.
One of its uses is illustrated by the following lemma, which follows
directly from the definition of shifted degree. %~\cite[Lemma~3.1]{ZLS2012}.% and \citep[Corollary 2.5]{zhou:phd2012}. 

\begin{lem}
\label{lem:productDegreeBound}If the $\vec{u}$-column degrees of
$\mathbf{A}\in\mathbb{K}\left[x\right]^{m\times n}$ are bounded by
the corresponding entries of an integer list $\vec{v}\in\mathbb{Z}^{n}$,
(or equivalently, the $-\vec{v}$-row degrees of $\mathbf{A}$ are
bounded by $-\vec{u}$) and the $\vec{v}$-column degrees of $\mathbf{B}\in\mathbb{K}\left[x\right]^{n\times k}$
are bounded by $\vec{w}\in\mathbb{Z}^{k}$, then the $\vec{u}$-column
degrees of $\mathbf{A}\mathbf{B}$ are bounded by $\vec{w}$. 
\end{lem}
\begin{comment}
A closely related result involving the shifted degrees is the following
lemma~\citep[Lemma 2.19]{zhou:phd2012}, which can be viewed as a
stronger version of the \emph{predictable degree property}~\citep[Theorem~6.3-13]{Kailath:1980}. 
\begin{lem}
\label{lem:predictableDegree}Let $\mathbf{A}\in\mathbb{K}[x]^{\ast\times m}$
be a $\vec{s}$-column reduced matrix with no zero columns and with
$\cdeg_{\vec{s}}\mathbf{A}=\vec{t}$. Then any matrix $\mathbf{B}\in\mathbb{K}[x]^{m\times\ast}$
satisfies $\cdeg_{\vec{t}}\mathbf{B}=\cdeg_{\vec{s}}\left(\mathbf{A}\cdot\mathbf{B}\right)$. \end{lem}
\end{comment}


An essential subroutine needed in our algorithm, also based on the
use of the shifted degrees, is the efficient multiplication of a pair
of matrices $\mathbf{A}\cdot\mathbf{B}$ with unbalanced degrees%
\begin{comment}
. The following result follows from 
\end{comment}
{} \citep[Theorem 5.6]{zhou:phd2012}. The notation $\sum\vec{s}$,
for any list $\vec{s}$, denotes the sum of all entries in $\vec{s}$. 
\begin{thm}
\label{thm:multiplyUnbalancedMatrices} %
\begin{comment}
Let $\mathbf{A}\in\mathbb{K}[x]^{n\times m}$ and $\mathbf{B}\in\mathbb{K}[x]^{m\times m}$
be given. Suppose $\vec{s}\in\mathbb{Z}_{\geq0}^{n}$ is a shift that
bounds the corresponding column degrees of $\mathbf{A}$, and $\sum\vec{s}\ge\sum\cdeg_{\vec{s}}\mathbf{B}$.
Then the product $\mathbf{A}\cdot\mathbf{B}$ can be computed in $O^{\sim}(n^{\omega}\hat{s})$
field operations from $\mathbb{K}$, where $\hat{s}=\sum\vec{s}/n$
is the average of the entries of $\vec{s}$. 
\end{comment}
Let $\mathbf{A}\in\mathbb{K}\left[x\right]^{m\times n}$ with $m\le n$,
$\vec{s}\in\mathbb{Z}^{n}$ a shift with entries bounding the column
degrees of $\mathbf{A}$ and $\check{\xi}$, a bound on the sum of
the entries of $\vec{s}$. Let $\mathbf{B}\in\mathbb{K}\left[x\right]^{n\times k}$
with $k\in O\left(m\right)$ and the sum $\theta$ of its $\vec{s}$-column
degrees satisfying $\theta\in O\left(\check{\xi}\right)$. Then we
can multiply $\mathbf{A}$ and $\mathbf{B}$ with a cost of $O^{\sim}(n{}^{\omega}t)$,
where $t=\check{\xi}/n$. 
\end{thm}

\subsection{Kernel bases and column bases}

Let $\mathbf{F}\in\mathbb{K}\left[x\right]^{m\times n}$ be a matrix
of polynomials over a field $\mathbb{K}$. The kernel of $\mathbf{F}\in\mathbb{K}\left[x\right]^{m\times n}$
is the $\mathbb{K}\left[x\right]$-module 
\[
\left\{ \mathbf{p}\in\mathbb{K}\left[x\right]^{n}\mid\mathbf{F}\cdot\mathbf{p}=0\right\} 
\]
with a kernel basis of $\mathbf{F}$ being a basis of this module. 
\begin{defn}
Given $\mathbf{F}\in\mathbb{K}\left[x\right]^{m\times n}$, a polynomial
matrix $\mathbf{N}\in\mathbb{K}\left[x\right]^{n\times*}$ is a $\vec{s}$-minimal
(right) kernel basis of $\mathbf{F}$ if\textbf{ $\mathbf{N}$} is
a kernel basis of $\mathbf{F}$ and $\mathbf{N}$ is $\vec{s}$-column
reduced. We also call a $\vec{s}$-minimal (right) kernel basis of
$\mathbf{F}$ a $\left(\mathbf{F},\vec{s}\right)$-kernel basis. 
\end{defn}
In this paper we require two essential properties of shifted minimal
kernel bases : a bound on the size of the minimal kernel and the cost
of computing such a minimal kernel. Both results come from \citep{za2012}. 
\begin{thm}
\label{thm:boundOfSumOfShiftedDegreesOfKernelBasis} Suppose $\mathbf{F}\in\mathbb{K}\left[x\right]^{m\times n}$
and $\vec{s}\in\mathbb{Z}_{\ge0}^{n}$ is a shift with entries bounding
the corresponding column degrees of $\mathbf{F}$. Then the sum of
the $\vec{s}$-column degrees of any $\vec{s}$-minimal kernel basis
of $\mathbf{F}$ is bounded by $\xi=\sum\vec{s}$. 
\end{thm}

\begin{thm}
\label{thm:kerncomp} Let $\mathbf{F}\in\mathbb{K}^{m\times n}$ and
$\vec{s}$ be a shift bounding the corresponding column degrees of
$\mathbf{F}$. Then a $\vec{s}$-minimal kernel basis of $\mathbf{F}$
can be computed in $O^{\sim}\left(n^{\omega}s\right)$ field operations
from $\mathbb{K}$, where $s$ is the average of the $m$ largest
column degrees of $\mathbf{F}$. 
\end{thm}
A column basis of $\mathbf{F}$ is a basis for the $\mathbb{K}\left[x\right]$-module
\[
\left\{ \mathbf{F}\mathbf{p}~|~\mathbf{p}\in\mathbb{K}\left[x\right]^{n}~\right\} ~.
\]
Such a basis can be represented as a full rank matrix $\mathbf{T}\in\mathbb{K}\left[x\right]^{m\times r}$
whose columns are the basis elements. Equivalently, there exists a
unimodular matrix $\mathbf{U}$ that transforms the matrix $\mathbf{F}$
to $\mathbf{F}\mathbf{U}=\left[\mathbf{T},0\right]$ with a full rank
matrix $\mathbf{T}$ defined as a column basis of $\mathbf{F}$.

The cost of column basis computation for $\mathbf{F}\in\mathbb{K}[z]^{m\times n}$
is given in \citep{zl2013} with the cost given as %having complexity 
$\bigO^{\sim}\left(nm^{\omega-1}\hat{s}\right)$ field operations
in $\mathbb{K}$, where $\hat{s}$ is the average column degree of
$\mathbf{F}$.


\subsection{Order bases}

Let $\mathbb{K}$ be a field, $\mathbf{F}\in\mathbb{K}\left[x\right]^{m\times n}$
a matrix of polynomials and $\vec{\sigma}$ a list non-negative integer. 
\begin{defn}
A vector of polynomials $\mathbf{p}\in\mathbb{K}\left[x\right]^{n\times1}$
has \emph{order} $\left(\mathbf{F},\vec{\sigma}\right)$ (or \emph{order}
$\vec{\sigma}$ with respect to $\mathbf{F}$) if $\mathbf{F}\cdot\mathbf{p}\equiv\mathbf{0}\mod x^{\vec{\sigma}}$,
that is, 
\[
\mathbf{F}\cdot\mathbf{p}=x^{\vec{\sigma}}\mathbf{r}
\]
for some $\mathbf{r}\in\mathbb{K}\left[x\right]^{m\times1}$.

The set of all order $\left(\mathbf{F},\vec{\sigma}\right)$ vectors
is a $\mathbb{K}\left[x\right]$-module denoted by $\left\langle \left(\mathbf{F},\vec{\sigma}\right)\right\rangle $. 
\end{defn}
Note that the matrix $\mathbf{F}$ can also be a matrix of power series
\citep{BeLa94}, but restricting it to matrix of polynomials is sufficient
for our purpose in this paper. Also note that the order $\vec{\sigma}$
may not be uniform%
\begin{comment}
as in some other applications of order basis, so the order of different
rows may be different 
\end{comment}
.

An \emph{order basis} $\mathbf{P}$ of $\mathbf{F}$ with order $\vec{\sigma}$
and shift $\vec{s}$, or an $\left(\mathbf{F},\vec{\sigma},\vec{s}\right)$-order
basis, or simply an $\left(\mathbf{F},\vec{\sigma},\vec{s}\right)$-basis,
is a polynomial matrix whose columns form a basis for the module $\left\langle \left(\mathbf{F},\vec{\sigma}\right)\right\rangle $
having minimal $\vec{s}$-column degrees \citep{BeLa94,BL1997}. Again,
note that a $\vec{s}$-column reduced basis of $\left\langle \left(\mathbf{F},\vec{\sigma}\right)\right\rangle $
has the minimal $\vec{s}$-column degrees among all bases of $\left\langle \left(\mathbf{F},\vec{\sigma}\right)\right\rangle $.
\begin{comment}
\begin{defn}
A polynomial matrix $\mathbf{P}$ is an order basis of $\mathbf{F}$
of order $\vec{\sigma}$ and shift $\vec{s}$, denoted by $\left(\mathbf{F},\sigma,\vec{s}\right)$-basis,
if the following properties hold: 
\begin{enumerate}
\item $\mathbf{P}$ is nonsingular and $\vec{s}$-column reduced. 
\item $\mathbf{P}$ has order $\left(\mathbf{F},\vec{\sigma}\right)$ (or
equivalently, each column of $\mathbf{P}$ is in $\left\langle (\mathbf{F},\vec{\sigma})\right\rangle $). 
\item Any $\mathbf{q}\in\left\langle \left(\mathbf{F},\vec{\sigma}\right)\right\rangle $
can be expressed as a linear combination of the columns of $\mathbf{P}$,
given by $\mathbf{P}^{-1}\mathbf{q}$. \end{enumerate}
\end{defn}
\end{comment}


We will compute order bases with unbalanced shift using Algorithm
2 from \citep{ZL2012} with the following cost. 
\begin{thm}
\label{thm:unbalancedOrderBasisCost}For an input matrix $\mathbf{F}\in\mathbb{K}\left[x\right]^{m\times n}$,
if the shift $\vec{s}$ satisfies $\vec{s}\ge0$ and $\sum\vec{s}\in O(m\sigma)$,
then a $\left(\mathbf{F},\sigma,-\vec{s}\right)$-basis can be computed
with a cost of $O^{\sim}(n^{\omega}a)$ field operations, where $a=m\sigma/n$. 
\end{thm}
We will need a special case of Theorem 5.1 in \citep{BL1997} which
for completeness is stated below. 
\begin{thm}
\label{thm:combineOrderBases} For a matrix $\mathbf{F}\in\mathbb{K}\left[x\right]^{m\times n}$,
an order vector $\vec{\sigma}$, and a shift vector $\vec{s}$, if
$\mathbf{P}$ is a $\left(\mathbf{F},\vec{\sigma},\vec{s}\right)$-basis
with $\vec{s}$-column degrees $\vec{t}$, and $\mathbf{Q}$ is a
$(\mathbf{F}\mathbf{P},\vec{\tau},\vec{t})$ -basis with $\vec{t}$-column
degrees $\vec{u}$, where $\vec{\tau}\ge\vec{\sigma}$ component-wise,
then $\mathbf{P}\mathbf{Q}$ is a $(\mathbf{F},\vec{\tau},\vec{s})$-basis
with $\vec{s}$-column degrees $\vec{u}$. 
\end{thm}
\begin{comment}
From \citep{BL1997} we have the following lemma. 
\begin{lem}
\label{lem:orderBasisProperty}

\label{lem:orderBasisEquivalence}The following are equivalent for
a polynomial matrix \textbf{$\mathbf{P}$}: 
\begin{enumerate}
\item $\mathbf{P}$ is a $\left(\mathbf{F},\vec{\sigma},\vec{s}\right)$-basis. 
\item $\mathbf{P}$ is comprised of a set of $n$ minimal $\vec{s}$-degree
polynomial vectors that are linearly independent and each having order
$\left(\mathbf{F},\vec{\sigma}\right)$. 
\item \label{enu:reduced+generator}$\mathbf{P}$ does not contain a zero
column, has order $\left(\mathbf{F},\vec{\sigma}\right)$, is $\vec{s}$-column
reduced, and any $\mathbf{q}\in\left\langle \left(\mathbf{F},\vec{\sigma}\right)\right\rangle $
can be expressed as a linear combination of the columns of $\mathbf{P}$. \end{enumerate}
\end{lem}
\end{comment}


In this paper we also will need the following lemma from \citep{zl2013,zhou:phd2012}
which shows that a left kernel basis of a right kernel basis is contained
in an order basis of a right kernel basis. 
\begin{lem}
\label{lem:nullspaceBasisInOrderBasis} For a matrix $\mathbf{A}\in\mathbb{K}\left[x\right]^{m\times n}$,
and a shift vector $\vec{s}$, if $\mathbf{N}$ is a $(\mathbf{A},\vec{s})$-kernel
basis with $\cdeg_{\vec{s}}\mathbf{N}=\vec{b}$ and $\mathbf{P}$
be a $\left(\mathbf{N}^{T},\vec{b}+1,-\vec{s}\right)$-basis. Partition
$\mathbf{P}=\left[\mathbf{P}_{1},\mathbf{P}_{2}\right]$ where $\mathbf{P}_{1}$
consists of all columns $\mathbf{p}$ with $\cdeg_{-\vec{s}~}\mathbf{p}\le0$.
Then $\mathbf{P}_{1}$ is a $(\mathbf{N}^{T},-\vec{s})$-kernel basis. 
\end{lem}
We remark that the condition $\cdeg_{-\vec{s}~}\mathbf{p}\le0$ is
the same as specifying $\deg p_{i}\leq s_{i}$ for all $i$. %\input{rest}



\subsection{The existence of unimodular completion}

We know that row vectors whose entries are relatively prime can be
completed to unimodular matrices. We also have a more general criterion
on the existence of unimodular completion for matrices. 
\begin{lem}
\label{lem:unimodularCompletionCondition}A unimodular completion
of $\mathbf{F}$ exists if and only if $\mathbf{F}$ has unimodular
column bases. \end{lem}
\begin{proof}
If $\mathbf{F}$ has a non-unimodular column basis $\mathbf{A}$,
then $\diag\left(\left[\mathbf{A},I\right]\right)$ is always a factor
of $\begin{bmatrix}\mathbf{F}\\
\mathbf{B}
\end{bmatrix}$ for any polynomial matrix $\mathbf{B}$, implying that the matrix
$\begin{bmatrix}\mathbf{F}\\
\mathbf{B}
\end{bmatrix}$ is non-unimodular. On the other hand, if $\mathbf{F}$ has a unimodular
column basis, then recall that there exists a unimodular matrix $\mathbf{U}$
such that $\mathbf{F}\mathbf{U}=\left[I_{m},0\right]$, where $I_{m}$
is a column basis of $\mathbf{F}$. This gives $\mathbf{F}=\left[I_{m},0\right]\mathbf{U}^{-1}$
after rearranging, that is, $\mathbf{F}$ must be consists of the
top $m$ rows of $\mathbf{U}^{-1}$. The matrix $\mathbf{U}^{-1}$
is therefore a unimodular completion of the matrix $\mathbf{F}$. 
\end{proof}
The proof of \prettyref{lem:unimodularCompletionCondition} shows
that a unimodular completion of $\mathbf{F}$ can be obtained from
the unimodular matrix $\mathbf{U}$ that transforms $\mathbf{F}$
to its column bases. However, we may not be able to compute this $\mathbf{U}$
efficiently since its degree might be too large. More specifically,
$\mathbf{U}$ contains a kernel basis of $\mathbf{F}$ that may have
degree $\xi=\sum\vec{s}$, while each of the remaining columns of
$\mathbf{U}$ may also have degree $\xi$.


\section{Reversing Operations}

Our procedure relies heavily on operations that reverse the order
of coefficients of polynomial matrices. Reverse operations have been
used in the past for computing matrix normal forms using order basis
computations. Here we extend the reverse operations to work with shifted
degrees. We show how reverse operations can be applied systematically
to polynomial matrices with certain shifted degrees, and then provide
some properties of the reverse operations.

For a polynomial $p=p_{0}+p_{1}x+\dots+p_{u}x^{u}\in\mathbb{K}\left[x\right]$
with degree bounded by $u$, the order of its coefficients can be
reversed as 
\[
\rev(p,u)=p(x^{-1})\cdot x^{u}=p_{u}+p_{u-1}x+\cdots+p_{1}x^{u-1}+p_{0}x^{u}.
\]
This can be extended to column vectors and row vectors with shifted
degrees. 
\begin{defn}
Let $\vec{u}=\left[u_{1},\dots u_{n}\right]\in\mathbb{Z}^{n}$ be
a degree shift, and $\mathbf{a}=\left[a_{1},\dots,a_{n}\right]^{T}\in\mathbb{K}\left[x\right]^{n\times1}$,
a column vector with $\vec{u}$-column degree bounded by $v$. We
define 
\[
\colRev(\mathbf{a},\vec{u},v)=x^{-\vec{u}}\mathbf{a}(x^{-1})~x^{v}=\begin{bmatrix}\rev(a_{1},v-u_{1})\\
\vdots\\
\rev(a_{n},v-u_{n})
\end{bmatrix}.
\]
Similarly for a row vector $\mathbf{b}\in\mathbb{K}\left[x\right]^{1\times n}$
with $\rdeg_{\vec{u}}\mathbf{b}\le v$,%
\begin{comment}
$\vec{u}$-row degree bounded by $v$, 
\end{comment}
{} where $\vec{u}=\left[u_{1},\dots u_{n}\right]\in\mathbb{Z}^{n}$
is a degree shift, we define 
\[
\rowRev(\mathbf{b},\vec{u},v)=\colRev(\mathbf{b}^{T},\vec{u},v)^{T}=x^{v}\mathbf{b}(x^{-1})~x^{-\vec{u}}.
\]
\end{defn}
\begin{exmp}
If $\mathbf{f}=\left[10+x,5+x+2x^{2}\right]$, $\vec{u}=\left[-1,-2\right]$,
and $v=0$, then 
\begin{align*}
\rowRev(\mathbf{f},\vec{u},v) & =x^{0}\left[10+x^{-1},5+x^{-1}+2x^{-2}\right]\begin{bmatrix}x\\
 & x^{2}
\end{bmatrix}\\
 & =\left[10x+1,5x^{2}+x+2\right].
\end{align*}

\end{exmp}
We can extend the reverse operation further to polynomial matrices. 
\begin{defn}
Let $\vec{u}=\left[u_{1},\dots u_{n}\right]\in\mathbb{Z}^{n}$ be
a degree shift and $\mathbf{A}\in\mathbb{K}\left[x\right]^{n\times k}$
with $\vec{u}$-column degrees bounded by $\vec{v}=\left[v_{1},\dots,v_{k}\right]\in\mathbb{Z}^{k}$.
Define 
\begin{align*}
\colRev(\mathbf{A},\vec{u},\vec{v}) & =\left[\colRev(\mathbf{A},\vec{u},v_{1}),\dots,\colRev(\mathbf{A},\vec{u},v_{k})\right]\\
 & =x^{-\vec{u}}\mathbf{A}(x^{-1})~x^{\vec{v}}.
\end{align*}
Similarly, for $\vec{u}\in\mathbb{Z}^{n}$ and $\mathbf{B}\in\mathbb{K}\left[x\right]^{k\times n}$
with $\vec{u}$-row degrees bounded component-wise by $\vec{v}\in\mathbb{Z}^{k}$,
\begin{align*}
\rowRev(\mathbf{B},\vec{u},\vec{v}) & =\begin{bmatrix}\rowRev(\mathbf{B},\vec{u},v_{1})\\
\vdots\\
\rowRev(\mathbf{B},\vec{u},v_{k})
\end{bmatrix}\\
 & =\colRev(\mathbf{B}^{T},\vec{u},\vec{v})^{T}\\
 & =x^{\vec{v}}\mathbf{B}(x^{-1})~x^{-\vec{u}}.
\end{align*}
It is not difficult to see that 
\[
\rowRev(\mathbf{B},\vec{u},\vec{v})=\colRev(\mathbf{B},-\vec{v},-\vec{u}).
\]

\end{defn}
It is useful to note that any degree bound remains the same after
the reverse operations. 
\begin{lem}
\label{lem:reversedDegreeBound}If $\mathbf{A}\in\mathbb{K}\left[x\right]^{n\times k}$
has $\cdeg_{\vec{u}}\mathbf{A}\le\vec{v}$%
\begin{comment}
$\vec{u}$-column degrees bounded by the corresponding entries of
$\vec{v}$ 
\end{comment}
, then $\mathbf{A}^{c}=\colRev(\mathbf{A},\vec{u},\vec{v})$ also
has $\cdeg_{\vec{u}}\mathbf{A}^{c}\le\vec{v}$%
\begin{comment}
$\vec{u}$-column degrees bounded by the corresponding entries of
$\vec{v}$ 
\end{comment}
. 
\end{lem}
As one would expect, applying two reverse operations gives back the
original input. 
\begin{lem}
The following equalities holds:%
\begin{comment}
\begin{eqnarray*}
\rev\left(\rev(p,u),u\right) & = & p\\
\colRev\left(\colRev(\mathbf{a},\vec{u},v),\vec{u},v\right) & = & \mathbf{a}\\
\rowRev\left(\rowRev(\mathbf{b},\vec{u},v),\vec{u},v\right) & = & \mathbf{b}
\end{eqnarray*}
\end{comment}
\begin{eqnarray*}
\colRev\left(\colRev(\mathbf{A},\vec{u},\vec{v}),\vec{u},\vec{v}\right) & = & \mathbf{A}\\
\rowRev\left(\rowRev(\mathbf{B},\vec{u},\vec{v}),\vec{u},\vec{v}\right) & = & \mathbf{B}.
\end{eqnarray*}

\end{lem}
The following lemmas show the commutativity between reverse operations
and multiplications. 
\begin{lem}
\label{lem:reverseMatrixProduct}If $\mathbf{A}\in\mathbb{K}\left[x\right]^{m\times n}$
has $\cdeg_{\vec{u}}\mathbf{A}\le\vec{v}$%
\begin{comment}
has $\vec{u}$-column degrees bounded by $\vec{v}$ 
\end{comment}
, and $\mathbf{B}\in\mathbb{K}\left[x\right]^{n\times k}$ has $\cdeg_{\vec{v}}\mathbf{B}\le\vec{w}$,%
\begin{comment}
$\vec{v}$-column degrees bounded by $\vec{w}$, 
\end{comment}
{} then 
\[
\colRev(\mathbf{A},\vec{u},\vec{v})\colRev(\mathbf{B},\vec{v},\vec{w})=\colRev(\mathbf{A}\mathbf{B},\vec{u},\vec{w})
\]
has $\vec{u}$-column degrees bounded by $\vec{w}$. 
\end{lem}

\begin{lem}
\label{lem:reverseMatrixProduct2}If $\mathbf{A}\in\mathbb{K}\left[x\right]^{m\times n}$
has $\rdeg_{\vec{u}}\mathbf{A}\le\vec{v}$%
\begin{comment}
$\vec{u}$-row degrees bounded by $\vec{v}$ 
\end{comment}
, and $\mathbf{B}\in\mathbb{K}\left[x\right]^{n\times k}$ has $\cdeg_{-\vec{u}}\mathbf{B}\le\vec{w}$%
\begin{comment}
$-\vec{u}$-column degrees bounded by $\vec{w}$ 
\end{comment}
, then 
\[
\rowRev(\mathbf{A},\vec{u},\vec{v})\colRev(\mathbf{B},-\vec{u},\vec{w})=\colRev(\mathbf{A}\mathbf{B},-\vec{v},\vec{w}).
\]

\end{lem}
More details on these reverse operations are found in \citep{zhou:phd2012}.

There is a natural relationship between a shifted minimal kernel basis
and the reverse operation. 
\begin{lem}
\label{lem:reverseNullspaceBasis} Let $\vec{u}\in\mathbb{Z}^{n}$
and $\mathbf{A}\in\mathbb{K}\left[x\right]^{m\times n}$ with $(-\vec{u})$-row
degrees bounded by $\vec{a}$. If $\mathbf{A}^{r}=\rowRev\left(\mathbf{A},-\vec{u},\vec{a}\right)$
then a matrix $\mathbf{N}\in\mathbb{K}\left[x\right]^{n\times k}$with
$\vec{u}$-column degrees $\vec{b}$ is a $(\mathbf{A},\vec{u})$-kernel
basis if and only if $\mathbf{N}^{c}=\colRev\left(\mathbf{N},\vec{u},\vec{b}\right)$
is a $(\mathbf{A}^{r},\vec{u})$-kernel basis \end{lem}
\begin{proof}
First note that any vector $\mathbf{q}$ with $\cdeg_{\vec{u}}\mathbf{q}=\alpha$
is in the kernel of $\mathbf{A}$ if and only if $\mathbf{q}^{c}=\colRev\left(\mathbf{q},\vec{u},\alpha\right)$
is in the kernel of $\mathbf{A}^{r}$, since by \prettyref{lem:reverseMatrixProduct2}
$\mathbf{A}\cdot\mathbf{q}=0$ implies 
\[
\mathbf{A}^{r}\cdot\mathbf{q}^{c}=\colRev\left(\mathbf{A}\cdot\mathbf{q},-\vec{a},\alpha\right)=0,
\]
and $\mathbf{A}^{r}\cdot\mathbf{q}^{c}=0$ implies 
\[
\mathbf{A}\cdot\mathbf{q}=\colRev\left(\mathbf{A}^{r}\cdot\mathbf{q}^{c},-\vec{a},\alpha\right)=0.
\]
It follows that any matrix $\mathbf{N}$ is a kernel basis of $\mathbf{A}$
if and only if $\mathbf{N}^{c}=\colRev\left(\mathbf{N},\vec{u},\cdeg_{\vec{u}}\mathbf{N}\right)$
is a kernel basis of $\mathbf{A}^{r}$. \prettyref{lem:reversedDegreeBound}
then ensures that the minimality also holds at the same time. 
\end{proof}

\section{Unimodular completion}

In this section, we look at how a unimodular completion can be done
using a combination of kernel basis computations, order basis computations,
and reverse operations.

\begin{comment}
Let us look informally to see how all these operations enter into
the unimodular completion problem. Recall from equation (\ref{one})
that if $\mathbf{U}$ is a unimodular matrix such that $\mathbf{F}\cdot\mathbf{U}=[I_{m},0]$
then the inverse $\mathbf{V}$ is a unimodular completion with $\mathbf{F}$
being its top $m$ rows. Suppose now that $\mathbf{\hat{V}}$ denotes
the bottom $n-m$ rows of $\mathbf{V}$ and that $\mathbf{\hat{U}}$
denote the last $n-m$ columns of $\mathbf{U}$. Then 
\begin{equation}
\mathbf{F}\cdot\mathbf{\hat{U}}=0~~~\mbox{ and }~~~\mathbf{\hat{V}}\cdot\mathbf{\hat{U}}=I_{n-m}.\label{two}
\end{equation}
In particular $\mathbf{\hat{U}}$ is a kernel basis for $\mathbf{F}$.
For the second equation it is helpful to consider reversing coefficients
since in this case one obtains part of an order basis. That is, if
\begin{equation}
\mathbf{F}^{r}=\rowRev(\mathbf{F},-\vec{s},0),~~~\mathbf{\hat{V}}^{r}=\rowRev(\mathbf{\hat{V}},-\vec{v},0)~~~\mbox{ and }~~~\mathbf{\hat{U}}^{c}=\colRev(\mathbf{\hat{U}},-\vec{v},0)\label{four}
\end{equation}
then 
\begin{equation}
\mathbf{F}^{r}\cdot\mathbf{\hat{U}}^{c}=0~~~\mbox{ and }~~~\mathbf{\hat{V}}^{r}\cdot\mathbf{\hat{U}}^{c}=x^{\vec{u}+\vec{v}}.\label{three}
\end{equation}
Thus, we now have that $\mathbf{\hat{U}}^{c}$ is a kernel basis for
$\mathbf{F}^{r}$ and, taking transposes of the rightmost equation
in (\ref{three}), we see that the transpose of the desired completion
$\mathbf{\hat{V}}^{r}$ is a set of $n-m$ columns of an order basis
for the transpose of $\mathbf{\hat{U}}^{c}$.

Working more formally, we need to be more specific about have the
matrix polynomials have their coefficients reversed (i.e. how to specify
$\vec{s}$, $\vec{u}$ and $\vec{v}$ in equation (\ref{four})) along
with details on locating the specific $n-m$ columns of an order basis. 
\end{comment}
We will show in \prettyref{lem:reverseOrderBasisToUnimodular} a close
relationship between order bases and unimodular matrices, namely,
suitable reverse operations can be applied to order bases to obtain
unimodular matrices. This suggests that the problem of finding a unimodular
completion of $\mathbf{F}$ is equivalent to finding some order basis
containing a reversed $\mathbf{F}$. In addition, \prettyref{lem:nullspaceBasisInOrderBasis}
shows how a kernel basis can be embedded in an order basis, that is,
if we can make the reversed $\mathbf{F}$ a kernel basis of some matrix
$\mathbf{M}$, then there is an order basis of $\mathbf{M}$ that
contains the reversed $\mathbf{F}$. A natural choice for $\mathbf{M}$
is a kernel basis of the reversed $\mathbf{F}$. We actually have
two choices here. We can either reverse the coefficients of $\mathbf{F}$,
as we do in \prettyref{thm:unimodularComputation}, or we can reverse
the coefficients of a kernel basis of $\mathbf{F}$.

First let us look at how an order basis can lead to a unimodular matrix. 
\begin{lem}
\label{lem:reverseOrderBasisToUnimodular} Given any matrix $\mathbf{A}\in\mathbb{K}\left[x\right]^{m\times n}$
and $\vec{u}\in\mathbb{Z}^{n}$ %=\left[u_{1},\dots u_{n}\right]\in\mathbb{Z}^{n}$
 a degree shift, if $\mathbf{P}$ is an $\left(\mathbf{A},\vec{\sigma},\vec{u}\right)$-basis
with $\cdeg_{\vec{u}}\mathbf{P}=\vec{v}$. %=\left[v_{1},\dots,v_{k}\right]$. 
Then $\mathbf{P}^{c}=\colRev(\mathbf{P},\vec{u},\vec{v})$ is unimodular. 
\end{lem}
%From the definition of $\mathbf{P}^c$ we see that we need only show that 

\begin{proof}
Note first that the identity matrix is an $\left(\mathbf{A},0,\vec{u}\right)$-basis,
which has $\vec{u}$-column degrees $\vec{u}$ and determinant $1$.
An $\left(\mathbf{A},\vec{\sigma},\vec{u}\right)$-basis $\mathbf{Q}$
can then be constructed iteratively based on \prettyref{thm:combineOrderBases}
and using the algorithms from \citep{BeLa94,Giorgi2003}, which only
increase the $\vec{u}$-column degrees of the basis by multiplying
some columns by $x$ each time. The minimality of order bases ensures
that $\cdeg_{\vec{u}}\mathbf{Q}=\cdeg_{\vec{u}}\mathbf{P}=\vec{v}$.
As the $\vec{u}$-column degrees of the basis are increased from $\vec{u}$
to $\vec{v}$, its determinant of $\mathbf{Q}$ therefore increased
from 1 to $x^{\sum\vec{v}-\sum\vec{u}}$. Since any two $\left(\mathbf{A},\vec{\sigma},\vec{u}\right)$-bases
are unimodularly equivalent, we get 
\[
\det\left(\mathbf{P}\right)=\det\left(\mathbf{Q}\mathbf{U}\right)=c\cdot x^{\sum\vec{v}-\sum\vec{u}}
\]
for some unimodular matrix $\mathbf{U}\in\mathbb{K}\left[x\right]^{n\times n}$
and nonzero constant $c\in\mathbb{K}$. Hence 
\begin{eqnarray*}
\det\left(\mathbf{P}^{c}\right) & = & \det\left(x^{-\vec{u}}\left(\mathbf{P}(x^{-1})\right)x^{\vec{v}}\right)\\
 & = & x^{-\sum\vec{u}}\cdot\det\left(\mathbf{P}(x^{-1})\right)\cdot x^{\sum\vec{v}}\\
 & = & c\cdot x^{\sum\vec{v}}\cdot x^{-\sum\vec{v}+\sum\vec{u}}\cdot x^{-\sum\vec{u}}=c
\end{eqnarray*}
and $\mathbf{P}^{c}$ is unimodular. \end{proof}
\begin{exmp}
\label{example:5} Let $\mathbf{A}\in\Z_{5}[x]^{2\times4}$ be given
by%
\begin{comment}
\[
\left[\begin{array}{cccc}
0 & -{x}^{2} & 2\,{x}^{4}-1 & 2\,{x}^{4}+x-1\\
\noalign{\medskip}-2\,{x}^{3} & {x}^{2} & -{x}^{4}+{x}^{3}+1 & -{x}^{3}-x+1
\end{array}\right]
\]
\end{comment}
\[
\left[\begin{array}{cccc}
1 & 0 & 2+2\, x & 3+4\, x\\
\noalign{\medskip}0 & {x}^{2} & 1+3\,{x}^{4} & 1+4\, x+3\,{x}^{4}
\end{array}\right]
\]
and $\vec{u}=[-2,-3,-1,-1]$ . Then one can show that%
\begin{comment}
\[
\mathbf{P}=\left[\begin{array}{cccc}
{x}^{2}-2\, x & 2\, x-1 & -x & -x\\
\noalign{\medskip}2 & 2\,{x}^{3}-1 & -{x}^{3}+1 & 1\\
\noalign{\medskip}-2\, x & 2\, x-1 & -x & -x\\
\noalign{\medskip}2\, x & -x+1 & x & x
\end{array}\right]
\]
\end{comment}
\[
\mathbf{P}=\left[\begin{array}{cccc}
4\, x+3\,{x}^{2} & 4+x+3\,{x}^{2} & {x}^{3} & 0\\
\noalign{\medskip}1 & 2\,{x}^{3} & 0 & {x}^{4}\\
\noalign{\medskip}4\, x & 4+x & 0 & 0\\
\noalign{\medskip}x & 1 & 0 & 0
\end{array}\right]
\]
is an $\left(\mathbf{A},[3,6],\vec{u}\right)$-basis with $\cdeg_{\vec{u}}\mathbf{P}=[0,0,1,1]$.
Reversing coefficients gives the unimodular matrix%
\begin{comment}
\[
\mathbf{P}^{c}=\left[\begin{array}{cccc}
-2\, x+1 & -{x}^{2}+2\, x & -x & -x\\
\noalign{\medskip}2\,{x}^{3} & -{x}^{3}+2 & {x}^{3}-1 & {x}^{3}\\
\noalign{\medskip}-2 & -x+2 & -1 & -1\\
\noalign{\medskip}2 & x-1 & 1 & 1
\end{array}\right].
\]
\end{comment}
\[
\mathbf{P}^{c}=\left[\begin{array}{cccc}
4\, x+3 & 4\,{x}^{2}+x+3 & 1 & 0\\
\noalign{\medskip}{x}^{3} & 2 & 0 & 1\\
\noalign{\medskip}4 & 4\, x+1 & 0 & 0\\
\noalign{\medskip}1 & x & 0 & 0
\end{array}\right].
\]
{} \qed 
\end{exmp}
The following lemma shows the unimodular equivalence between any matrix
$\mathbf{A}$ that has a unimodular column basis, and a left kernel
basis of any right kernel basis of $\mathbf{A}$. This is needed later
in \prettyref{thm:unimodularComputation} in order to replace $\mathbf{F}$
with a kernel basis that can be embedded in a unimodular matrix. 
\begin{lem}
\label{lem:unimodularEquivalenceNullspaceBasisOfNullspaceBasis} Let
$\mathbf{A}\in\mathbb{K}\left[x\right]^{m\times n}$ have a unimodular
column basis and $\mathbf{N}\in\mathbb{K}\left[x\right]^{n\times\left(n-m\right)}$
be a right kernel basis of $\mathbf{A}$. If $\mathbf{B}$ is a left
kernel basis of $\mathbf{N}$ then $\mathbf{A}=\mathbf{U}\mathbf{B}$
for a unimodular matrix $\mathbf{U}$. \end{lem}
\begin{proof}
This follows from \citep[Lemma 3.3]{zl2013} which tells us that if
$\mathbf{U}$ is a column basis of $\mathbf{A}$ then $\mathbf{A}=\mathbf{U}\mathbf{B}$. 
\end{proof}
We are now ready to state a key result that allows a unimodular completion
to be computed for a given input matrix $\mathbf{F}$. Basically,
we compute a kernel basis of a reversed $\mathbf{F}$, then an order
basis of the kernel basis can be reversed to provide a unimodular
completion of $\mathbf{F}$. 
\begin{thm}
\label{thm:unimodularComputation} Let $\mathbf{F}^{r}=\rowRev\left(\mathbf{F},-\vec{u},0\right)$
and $\mathbf{M}$ be a $(\mathbf{F}^{r},\vec{u})$-kernel basis with
\begin{comment}
shifted column 
\end{comment}
$\cdeg_{\vec{u}}\mathbf{M}=\vec{b}$. Let $\mathbf{P}=\left[\mathbf{P}_{1},\mathbf{P}_{2}\right]$
be a $\left(\mathbf{M}^{T},\vec{b}+1,-\vec{u}\right)$-basis, where
$\mathbf{P}_{1}$ consists of all columns $\mathbf{p}$ with $\cdeg_{-\vec{u}}\mathbf{p}\le0$.
If $\mathbf{P}_{2}^{r}=\colRev\left(\mathbf{P}_{2},-\vec{u},\cdeg_{-\vec{u}}\mathbf{P}_{2}\right)$,
then $\left[\begin{array}{c}
\mathbf{F}\\
{\mathbf{P}_{2}^{r}}^{T}
\end{array}\right]$ is a unimodular completion of $\mathbf{F}$. \end{thm}
\begin{proof}
Let $\mathbf{P}_{1}^{r}=\colRev\left(\mathbf{P}_{1},-\vec{u},\vec{v}\right)$
where $\vec{v}=\cdeg_{-\vec{u}}\mathbf{P}_{1}$ is the shifted degree
of $\mathbf{P}_{1}$. We know from \prettyref{lem:reverseOrderBasisToUnimodular}
that $\left[\mathbf{P}_{1}^{r},\mathbf{P}_{2}^{r}\right]$ is unimodular.
Let $\mathbf{M}^{r}=\colRev\left(\mathbf{M},\vec{u},\vec{b}\right)$.
Then from \prettyref{lem:reverseNullspaceBasis} we know $\mathbf{M}^{r}$
is a $\left(\mathbf{F},\vec{s}\right)$-kernel basis and $\mathbf{P}_{1}^{r}$
is a $\left(\left(\mathbf{M}^{r}\right)^{T},-\vec{u}\right)$-kernel
basis. Hence, by \prettyref{lem:unimodularEquivalenceNullspaceBasisOfNullspaceBasis},
$\mathbf{F}=\mathbf{U}\left(\mathbf{P}_{1}^{r}\right)^{T}$ for some
unimodular matrix $\mathbf{U}$. Therefore 
\[
\left[\begin{array}{c}
\mathbf{F}\\
{\mathbf{P}_{2}^{r}}^{T}
\end{array}\right]=\left[\begin{array}{cc}
\mathbf{U} & 0\\
0 & I
\end{array}\right]\cdot\left[\begin{array}{c}
{\mathbf{P}_{1}^{r}}^{T}\\
{\mathbf{P}_{2}^{r}}^{T}
\end{array}\right]
\]
%a product of two unimodular matrices and 
hence is itself unimodular. \end{proof}
\begin{exmp}
Let $\mathbf{F}\in\Z_{5}[x]^{2\times4}$ be given by 
\[
\left[\begin{array}{cccc}
-2\, x+1 & 2\,{x}^{3} & -2 & 2\\
\noalign{\medskip}-{x}^{2}+2\, x & -{x}^{3}+2 & -x+2 & x-1
\end{array}\right]
\]
with shift $\vec{u}=[2,3,1,1]$. Then the transpose of a $\vec{u}$-minimal
kernel basis for the row reversed polynomial matrix, $\mathbf{F}^{r}$,
is given by%
\begin{comment}
\[
\mathbf{M}=\left[\begin{array}{cc}
0 & -2\,{x}^{3}\\
\noalign{\medskip}-{x}^{2} & {x}^{2}\\
\noalign{\medskip}2\,{x}^{4}-1 & -{x}^{4}+{x}^{3}+1\\
\noalign{\medskip}2\,{x}^{4}+x-1 & -{x}^{3}-x+1
\end{array}\right]
\]
\end{comment}


\[
\mathbf{M}^{T}=\left[\begin{array}{cccc}
1 & 0 & 2+2\, x & 4\, x+3\\
\noalign{\medskip}0 & {x}^{2} & 1+3\,{x}^{4} & 1+4\, x+3\,{x}^{4}
\end{array}\right]
\]
with $\vec{b}=\cdeg_{\vec{u}}\mathbf{M}=[2,5]$. An $(\mathbf{M}^{T},\vec{b}+1,-\vec{u})$-order
basis is given by%
\begin{comment}
\[
\mathbf{P}=\left[\begin{array}{cccc}
{x}^{2}-2\, x & 2\, x-1 & -x & -x\\
\noalign{\medskip}2 & 2\,{x}^{3}-1 & -{x}^{3}+1 & 1\\
\noalign{\medskip}-2\, x & 2\, x-1 & -x & -x\\
\noalign{\medskip}2\, x & -x+1 & x & x
\end{array}\right]
\]
\end{comment}


\[
\mathbf{P}=\left[\begin{array}{cccc}
4\, x+3\,{x}^{2} & 4+x+3\,{x}^{2} & {x}^{3} & 0\\
\noalign{\medskip}1 & 2\,{x}^{3} & 0 & {x}^{4}\\
\noalign{\medskip}4\, x & 4+x & 0 & 0\\
\noalign{\medskip}x & 1 & 0 & 0
\end{array}\right]
\]
from Example \ref{example:5}. Reversing the last two columns of $\mathbf{P}$
and taking transposes give a unimodular completion%
\begin{comment}
\[
{\mathbf{P}_{2}^{r}}^{T}=\left[\begin{array}{cccc}
-x & {x}^{3}-1 & -1 & 1\\
\noalign{\medskip}-x & {x}^{3} & -1 & 1
\end{array}\right].
\]
\end{comment}
\[
\mathbf{P}_{2}^{rT}=\left[\begin{array}{cccc}
1 & 0 & 0 & 0\\
\noalign{\medskip}0 & 1 & 0 & 0
\end{array}\right].
\]
\qed 
\end{exmp}
\prettyref{thm:unimodularComputation} provides a way to correctly
compute a unimodular completion of $\mathbf{F}$. To improve the computational
efficiency, it is helpful to separate the rows of $\mathbf{M}^{T}$
and just work with one subset of rows at a time. This is possible
by the following lemma. 
\begin{lem}
\label{lem:unimodularComputationByRows} Let $\mathbf{F}^{r}=\rowRev\left(\mathbf{F},-\vec{u},0\right)$
and $\mathbf{M}$ be a $(\mathbf{F}^{r},\vec{u})$-kernel basis with
$\cdeg_{\vec{u}}\mathbf{M}=\vec{b}$ and partitioned as $\mathbf{M}=\left[\mathbf{M}_{1},\mathbf{M}_{2}\right]$.
Let $\mathbf{P}_{1}$ be a $\left(\mathbf{M}_{1}^{T},\cdeg_{\vec{u}}\mathbf{M}_{1}+1,-\vec{u}\right)$-basis
%and be 
partitioned as $\mathbf{P}_{1}=\left[\mathbf{N}_{1},\mathbf{Q}_{1}\right]$,
where $\mathbf{N}_{1}$ consists of all columns $\mathbf{p}$ of $\mathbf{P}_{1}$
with $\cdeg_{-\vec{u}}\mathbf{p}\le0$. Let $\vec{t}=\cdeg_{-\vec{u}}\mathbf{N}_{1}$
and $\mathbf{P}_{2}$ be a $\left(\mathbf{M}_{2}^{T}\mathbf{N}_{1},\cdeg_{\vec{u}}\mathbf{M}_{2}+1,\vec{t}\right)$-basis
%and be 
partitioned as $\mathbf{P}_{2}=\left[\mathbf{N}_{2},\mathbf{Q}_{2}\right]$,
where $\mathbf{N}_{2}$ consists of all columns $\mathbf{p}$ of $\mathbf{P}_{2}$
with $\cdeg_{-\vec{t}}\mathbf{p}\le0$. If $\mathbf{R}=\left[\mathbf{N}_{1}\mathbf{Q}_{2},\mathbf{Q}_{1}\right]$
and $\mathbf{R}^{r}=\colRev\left(\mathbf{R},-\vec{u},\cdeg_{-\vec{u}}\mathbf{R}\right)$,
then $\left[\begin{array}{c}
\mathbf{F}\\
{\mathbf{R}^{r}}^{T}
\end{array}\right]$ is a unimodular completion of $\mathbf{F}$. \end{lem}
\begin{proof}
Let $\mathbf{P}_{1}^{r}=\colRev\left(\mathbf{P}_{1},-\vec{s},\cdeg_{-\vec{s}}\mathbf{P}_{1}\right)$
and $\mathbf{P}_{2}^{r}=\colRev\left(\mathbf{P}_{2},\vec{t},\cdeg_{\vec{t}}\mathbf{P}_{2}\right)$.
From \prettyref{lem:reverseOrderBasisToUnimodular} we have that both
$\mathbf{P}_{1}^{r}$ and $\mathbf{P}_{2}^{r}$ are unimodular and
hence 
\[
\mathbf{P}_{1}^{r}\cdot\diag\left(\left[\mathbf{P}_{2}^{r},I\right]\right)=\left[\mathbf{N}_{1}^{r}\mathbf{N}_{2}^{r},\mathbf{N}_{1}^{r}\mathbf{Q}_{2}^{r},\mathbf{Q}_{1}\right]=\left[\mathbf{N}_{1}^{r}\mathbf{N}_{2}^{r},\mathbf{R}^{r}\right]
\]
is unimodular, where\textbf{ $\mathbf{N}_{1}\mathbf{N}_{2}$ }is a
kernel basis of $\mathbf{M}$. The result follows by the same reasoning
as in the proof of \prettyref{thm:unimodularComputation}. 
\end{proof}

\section{Efficient Computation}

\prettyref{lem:unimodularComputationByRows} provides a way to correctly
compute a unimodular completion of $\mathbf{F}$. Our next task is
to make sure it can be computed efficiently and analyze its computational
cost. We already know that a $(\mathbf{F}^{r},\vec{s})$-kernel basis
can be computed with a cost of $O^{\sim}\left(n^{\omega}s\right)$.
Therefore, it only remains to check the cost of the order basis computations.
Note that the non-uniform order makes our problem here a little more
difficult. However, the output basis has its $-\vec{s}$-column degrees
bounded by $1$, which is a consequence of the fact $\mathbf{M}$
is a $\vec{s}$-minimal kernel basis, as shown in \prettyref{lem:nullspaceOrderbasisDegree}
below. But we first need a few general lemmas on the degree bounds
of order bases and kernel bases.

First, the following lemma is a simple extension of Lemma 3.2 in \citep{za2012}
for dealing with nonuniform orders. 
\begin{lem}
\label{lem:boundOfSumOfShiftedDegreesOfOrderBasisWithNonuniformOrder}
Given $\mathbf{A}\in\mathbb{K}^{m\times n}[x]$, a shift $\vec{u}\in\mathbb{Z}^{n}$,
and an order list $\vec{\sigma}\in\mathbb{Z}^{m}$, let $\vec{v}$
be the $\vec{u}$-column degrees of a $\left(\mathbf{A},\vec{\sigma},\vec{u}\right)$-basis.
Then 
\[
\sum\vec{v}~\le~\sum\vec{u}+\sum\vec{\sigma}.
\]
\end{lem}
\begin{proof}
The sum of the $\vec{u}$-column degrees is $\sum\vec{u}$ at order
$0$, since the identity matrix is a $\left(\mathbf{A},0,\vec{u}\right)$-basis.
This sum increases by $1$ for each order increase of each row. The
total number of order increases required for all rows is at most $\sum\vec{\sigma}$.
Note that from \prettyref{thm:combineOrderBases}, we can work with
just one row at a time to increase its order in the order basis computation. 
\end{proof}
The following lemma extends \prettyref{thm:boundOfSumOfShiftedDegreesOfKernelBasis}
to give a bound based on the shifted column degrees or shifted row
degrees, instead of just the column degrees of the input matrix. 
\begin{lem}
\label{lem:generaKernelBasisDegreeBound}If $\mathbf{A}\in\mathbb{K}^{m\times n}[x]$
has $\rdeg_{\vec{u}}\mathbf{A}\le\vec{v}$ or equivalently $\cdeg_{-\vec{v}}\mathbf{A}\le-\vec{u}$,
then any $(\mathbf{A},-\vec{u})$-kernel basis $\mathbf{B}$ satisfies
\[
\sum\cdeg_{-\vec{u}}\mathbf{B}\le\sum\vec{v}-\sum\vec{u}.
\]
\end{lem}
\begin{proof}
Let $\mathbf{P}=\left[\mathbf{B},\bar{\mathbf{B}}\right]$ be a $\left(\mathbf{A},\vec{v}+\sigma,-\vec{u}\right)$-basis
containing a kernel basis, $\mathbf{B}$, of $\mathbf{A}$. Then 
\begin{equation}
\sum\cdeg_{-\vec{u}}\mathbf{P}\le m\sigma+\sum\vec{v}-\sum\vec{u}.\label{eq:orderBasisDegreeBound}
\end{equation}
by \prettyref{lem:boundOfSumOfShiftedDegreesOfOrderBasisWithNonuniformOrder}
\begin{comment}
This is because we can iteratively construct an $\left(\mathbf{A},\vec{v}+\sigma,-\vec{u}\right)$-basis
from the identity matrix, a $\left(\mathbf{A},0,-\vec{u}\right)$-basis
with $-\vec{u}$-column degrees $-\vec{u}$, by using at most 1 degree
increase for each increase in the order $\vec{v}+\sigma$. Then $\sum\cdeg_{-\vec{u}}\mathbf{P}$
is increased by at most $m\sigma+\sum\vec{v}$ from $-\sum\vec{u}$. 
\end{comment}
From \prettyref{lem:productDegreeBound} we also know that 
\[
\sum\cdeg_{-\vec{u}}\bar{\mathbf{B}}\ge\sum\cdeg_{-\vec{v}}\mathbf{A}\bar{\mathbf{B}}.
\]
% by \prettyref{lem:productDegreeBound}. 
In addition, we know 
\[
\cdeg\mathbf{A}\bar{\mathbf{B}}\ge\vec{v}+\sigma
\]
since $\bar{\mathbf{B}}$ has order $\left(\mathbf{A},\vec{v}+\sigma\right)$,
implying 
\[
\sum\cdeg_{-\vec{v}}\mathbf{A}\bar{\mathbf{B}}\ge m\sigma
\]
and so 
\begin{equation}
\sum\cdeg_{-\vec{u}}\bar{\mathbf{B}}\ge m\sigma.\label{eq:nonkernelBasisDegreeBound}
\end{equation}


Combining \prettyref{eq:orderBasisDegreeBound} and \prettyref{eq:nonkernelBasisDegreeBound},
It follows that 
\begin{eqnarray*}
\sum\cdeg_{-\vec{u}}\mathbf{B} & = & \sum\cdeg_{-\vec{u}}\mathbf{P}-\sum\cdeg_{-\vec{u}}\bar{\mathbf{B}}\\
 & \le & \sum\vec{v}-\sum\vec{u}.
\end{eqnarray*}

\end{proof}
Note that \prettyref{lem:generaKernelBasisDegreeBound} specializes
to \prettyref{thm:boundOfSumOfShiftedDegreesOfKernelBasis} when $\vec{v}=0$.

When the matrix $\mathbf{A}$ is also a $\left(\mathbf{B}^{T},\vec{u}\right)$-kernel
basis, as in our case, the bound in fact becomes tight. 
\begin{lem}
\label{lem:mutualMinimalNullspaceBasisDegrees}Let $\mathbf{A}\in\mathbb{K}^{m\times n}[x]$
and $\mathbf{B}\in\mathbb{K}^{n\times(n-m)}\left[x\right]$. If $\mathbf{B}$
is a $(\mathbf{A},-\vec{u})$-kernel basis with $\cdeg_{-\vec{u}}\mathbf{B}=\vec{w}$
and $\mathbf{A}^{T}$ is a $\left(\mathbf{B}^{T},\vec{u}\right)$-kernel
basis with $\rdeg_{\vec{u}}\mathbf{A}=\vec{v}$, then 
\[
\sum\vec{w}=\sum\vec{v}-\sum\vec{u}.
\]
\end{lem}
\begin{proof}
This follows from \prettyref{lem:generaKernelBasisDegreeBound}, which
gives 
\[
\sum\vec{w}\le\sum\vec{v}-\sum\vec{u}
\]
and also 
\[
\sum\vec{v}\le\sum\vec{w}+\sum\vec{u}
\]
in the reverse direction. 
\end{proof}
From \prettyref{lem:nullspaceBasisInOrderBasis}, we know that any
$\left(\mathbf{M}^{T},\vec{b}+1,-\vec{s}\right)$-basis contains a
$\left(\mathbf{M}^{T},-\vec{s}\right)$-kernel basis whose $-\vec{s}$-column
degrees are bounded by 0. The following lemma shows that the remaining
part of the $\left(\mathbf{M}^{T},\vec{b}+1,-\vec{s}\right)$-basis
has degrees bounded by 1. 
\begin{lem}
\label{lem:nullspaceOrderbasisDegree}Let $\mathbf{F}^{r}=\rowRev\left(\mathbf{F},-\vec{s},0\right)$
and $\mathbf{M}$ be a $(\mathbf{F}^{r},\vec{s})$-kernel basis with
$\cdeg_{\vec{s}}\mathbf{M}=\vec{b}$. %Let $\mathbf{P}$ be a $\left(\mathbf{M}^{T},\vec{b}+1,-\vec{s}\right)$-basis.
Partition $\mathbf{P}$, a \\
 $\left(\mathbf{M}^{T},\vec{b}+1,-\vec{s}\right)$-basis, as $\mathbf{P}=\left[\mathbf{P}_{1},\mathbf{P}_{2}\right]$
where where $\mathbf{P}_{1}$ consists of all columns $\mathbf{p}$
with $\cdeg_{-\vec{s}~}\mathbf{p}\le0$. Then 
\[
\cdeg_{-\vec{b}-1}\mathbf{M}^{T}\mathbf{P}_{2}=0~~\mbox{ and }~~\cdeg_{-\vec{s}}\mathbf{P}_{2}=1.
\]
\end{lem}
\begin{proof}
We already know that $\mathbf{P}$ contains a $\left(\mathbf{M}^{T},-\vec{s}\right)$-kernel
basis $\mathbf{P}_{1}$. Furthermore as in the proof of \prettyref{lem:generaKernelBasisDegreeBound}
we have 
\[
\sum\cdeg_{-\vec{s}}\mathbf{P}=-\sum\vec{s}+\sum\vec{b}+n-m
\]
while from \prettyref{lem:mutualMinimalNullspaceBasisDegrees} we
have for the kernel basis $\mathbf{P}_{1}$ in $\mathbf{P}$ 
\[
\sum\cdeg_{-\vec{s}}\mathbf{P}_{1}=\sum\vec{b}-\sum\vec{s}
\]
and therefore, $\sum\cdeg_{-\vec{s}}\mathbf{P}_{2}=n-m$. From \prettyref{lem:productDegreeBound},
it follows that 
\[
\sum\cdeg_{-\vec{b}}\mathbf{M}^{T}\mathbf{P}_{2}\le\sum\cdeg_{-\vec{s}}\mathbf{P}_{2}=n-m,
\]
or equivalently, 
\[
\sum\cdeg_{-\vec{b}-1}\mathbf{M}^{T}\mathbf{P}_{2}\le0.
\]
But since $\mathbf{P}_{2}$ is nonzero and has order $\left(\mathbf{F},\vec{b}+1\right)$,
we have 
\[
\sum\cdeg_{-\vec{b}-1}\mathbf{M}^{T}\mathbf{P}_{2}\ge0.
\]
It follows that 
\[
\sum\cdeg_{-\vec{b}-1}\mathbf{M}^{T}\mathbf{P}_{2}=0
\]
hence $\cdeg_{-\vec{b}-1}\mathbf{M}^{T}\mathbf{P}_{2}=0$ or $\cdeg_{-\vec{b}}\mathbf{M}^{T}\mathbf{P}_{2}=1$.
Combining this with 
\[
\sum\cdeg_{-\vec{b}}\mathbf{M}^{T}\mathbf{P}_{2}\le\sum\cdeg_{-\vec{s}}\mathbf{P}_{2}=n-m
\]
we then get $\cdeg_{-\vec{s}}\mathbf{P}_{2}=1.$ 
\end{proof}
We are now ready to look at the procedure for computing a $\left(\mathbf{M}^{T},\vec{b}+1,-\vec{s}\right)$-basis,
given in \prettyref{alg:unimodularCompletion}. The situation here
is similar to the situation in computing a left kernel basis in the
column basis computation from \citep{zl2013}. That is, the order
$\vec{b}+1$, or equivalently, the $\vec{s}$-row degrees of $\mathbf{M}^{T}$
may be unbalanced and can have degree as large as $\sum\vec{s}$.
We therefore follow the same process as in the computation of column
bases \citep{zl2013}.

\begin{comment}
If $k$ is the column dimension of $\mathbf{M}$ and $\xi$ is an
upper bound of $\sum\vec{s}$, then 
\[
\sum\cdeg_{\vec{s}}\mathbf{M}=\sum\vec{b}\le\sum\vec{s}\le\xi
\]
by \prettyref{thm:boundOfSumOfShiftedDegreesOfKernelBasis}, hence
at most $\frac{k}{c}$ columns of $\mathbf{M}$ have $\vec{s}$-column
degrees greater than or equal to $\frac{c~\xi}{k}$ for any $c\ge1$. 
\end{comment}
{} We assume without loss of generality that the rows of $\mathbf{M}^{T}$
are arranged in decreasing $\vec{s}$-row degrees and divide $\mathbf{M}^{T}$
into $\left\lceil \log k\right\rceil $ row blocks according to the
$\vec{s}$-row degrees of its rows. Let 
\[
\mathbf{M}^{T}=\begin{bmatrix}\mathbf{M}_{1}\\
\mathbf{M}_{2}\\
\vdots\\
\mathbf{M}_{\left\lceil \log k\right\rceil }
\end{bmatrix}^{T}
\]
with $\mathbf{M}_{\log k},\mathbf{M}_{\log k-1},\cdots,\mathbf{M}_{2},\mathbf{M}_{1}$
having $\vec{s}$-row degrees in the range 
\[
\left[0,2\xi/k\right],\ (2\xi/k,4\xi/k],\ (4\xi/k,8\xi/k],\ ...,\ (\xi/4,\xi/2],\ (\xi/2,\xi]
\]
respectively. Let $\sigma_{i}=\left\lceil \xi/2^{i-1}\right\rceil +1$
and $\vec{\sigma}_{i}=\left[\sigma_{i},\dots,\sigma_{i}\right]$ with
the same dimension as the row dimension of $\mathbf{M}_{i}$, and
\[
\vec{\sigma}=\left[\vec{\sigma}_{\left\lceil \log k\right\rceil },\vec{\sigma}_{\left\lceil \log k\right\rceil -1},\dots,\vec{\sigma}_{1}\right]
\]
be the order in the order basis computation. For simplicity, instead
of using $\mathbf{M}^{T}$ as the input matrix, we use 
\begin{eqnarray*}
\hat{\mathbf{M}} & =\begin{bmatrix}\hat{\mathbf{M}}_{1}\\
\vdots\\
\hat{\mathbf{M}}_{\left\lceil \log k\right\rceil }
\end{bmatrix}= & x^{\vec{\sigma}-\vec{b}-1}\begin{bmatrix}\mathbf{M}_{1}\\
\vdots\\
\mathbf{M}_{\left\lceil \log k\right\rceil }
\end{bmatrix}=x^{\vec{\sigma}-\vec{b}-1}\mathbf{M}^{T}
\end{eqnarray*}
instead, so that the order of our problem in each block is uniform
and a $\left(\hat{\mathbf{M}},\vec{\sigma},-\vec{s}\right)$-basis
is a $\left(\mathbf{M}^{T},\vec{b}+1,-\vec{s}\right)$-basis.

% missing paert removed


We now do a series of order basis computations in order to compute
a unimodular completion of $\mathbf{F}$ based on \prettyref{lem:unimodularComputationByRows}. 
\begin{enumerate}
\item Let $\vec{s}_{1}=\vec{s}$. First we compute an $\left(\hat{\mathbf{M}}_{1},\vec{\sigma}_{1},-\vec{s}_{1}\right)$-order
basis using Algorithm 2 of \citep{ZL2012}, which can be done with
a cost of $O^{\sim}\left(n^{\omega}s\right)$%
\begin{comment}
, where $s=\xi/n$ 
\end{comment}
. Partition $\mathbf{P}_{1}=\left[\mathbf{N}_{1},\mathbf{Q}_{1}\right]$,
where $\mathbf{N}_{1}$ is a $\left(\hat{\mathbf{M}}_{1},-\vec{s}_{1}\right)$-kernel
basis. Set $\tilde{\mathbf{N}}_{1}=\mathbf{N}_{1}$ and $\vec{s}_{2}=-\cdeg_{-\vec{s}}\mathbf{N}_{1}$. 
\item Compute an $\left(\hat{\mathbf{M}}_{2}\tilde{\mathbf{N}}_{1},\vec{\sigma}_{2},-\vec{s}_{2}\right)$-order
basis and partition $\mathbf{P}_{2}=\left[\mathbf{N}_{2},\mathbf{Q}_{2}\right]$
with $\mathbf{N}_{2}$ a $\left(\hat{\mathbf{M}}_{2},-\vec{s}_{2}\right)$-kernel
basis. Set $\tilde{\mathbf{N}}_{2}=\tilde{\mathbf{N}}_{1}\mathbf{N}_{2}$
and $\vec{s}_{3}=-\cdeg_{-\vec{s}_{2}}\mathbf{N}_{2}$. Note that
$\tilde{\mathbf{N}}_{2}$ is a $-\vec{s}$-minimal kernel basis of
$\begin{bmatrix}\hat{\mathbf{M}}_{1}\\
\hat{\mathbf{M}}_{2}
\end{bmatrix}$ with $\vec{s}_{3}=-\cdeg_{-\vec{s}}\tilde{\mathbf{N}}_{2}$. 
\item Continue this process, an $\left(\hat{\mathbf{M}}_{i}\tilde{\mathbf{N}}_{i-1},\vec{\sigma}_{i},-\vec{s}_{i}\right)$-order
basis $\mathbf{P}_{i}$ is computed at step $i$. Partition $\mathbf{P}_{i}=\left[\mathbf{N}_{i},\mathbf{Q}_{i}\right]$
with $\mathbf{N}_{i}$ an $\left(\hat{\mathbf{M}}_{i}\tilde{\mathbf{N}}_{i-1},-\vec{s}_{i}\right)$-kernel
basis, where $\vec{s}_{i}=-\cdeg_{-\vec{s}}\tilde{\mathbf{N}}_{i-1}=-\cdeg_{-\vec{s}_{i-1}}\mathbf{N}_{i-1}$.
Then 
\[
\tilde{\mathbf{N}}_{i}=\prod_{j=1}^{i}\mathbf{N}_{i}=\tilde{\mathbf{N}}_{i-1}\mathbf{N}_{i}
\]
is a $-\vec{s}$-minimal kernel basis of the first $i$ blocks of
$\mathbf{M}^{T}$. In particular, $\tilde{\mathbf{N}}_{\left\lceil \log k\right\rceil }$
is a $\left(\mathbf{M}^{T},-\vec{s}\right)$-kernel basis. 
\item Let $\mathbf{R}=$ 
\[
\left[\mathbf{Q}_{1},\tilde{\mathbf{N}}_{1}\mathbf{Q}_{2},\dots,\tilde{\mathbf{N}}_{\left\lceil \log k\right\rceil -2}\mathbf{Q}_{\left\lceil \log k\right\rceil -1},\tilde{\mathbf{N}}_{\left\lceil \log k\right\rceil -1}\mathbf{Q}_{\left\lceil \log k\right\rceil }\right],
\]
and $\mathbf{R}^{r}=\colRev\left(\mathbf{R},-\vec{s},\cdeg_{-\vec{s}}\mathbf{R}\right)$.
Then from \prettyref{lem:unimodularComputationByRows} we can conclude
that $\left[\mathbf{F}^{T},\mathbf{R}^{r}\right]$ is a unimodular
matrix. 
\end{enumerate}
\begin{algorithm}[t]
\caption{$\unimodularCompletion(\mathbf{F})$}
\label{alg:unimodularCompletion}

\begin{algorithmic}[1]
\REQUIRE{$\mathbf{F}\in\mathbb{K}\left[x\right]^{m\times n}$ with full row
rank; $\vec{s}$ is initially set to the column degrees of $\mathbf{F}$.
It keeps track of the degrees.}

\ENSURE{$\mathbf{G}\in\mathbb{K}\left[x\right]^{\left(n-m\right)\times n}$
such that $\begin{bmatrix}\mathbf{F}\\
\mathbf{G}
\end{bmatrix}$ is unimodular.}

\STATE{$\vec{s}:=\cdeg\mathbf{F}$;}

\STATE{$\mathbf{F}^{r}:=\rowRev\left(\mathbf{F},-\vec{s},0\right)$;}

\STATE{$\mathbf{M}:=\mnb(\mathbf{F}^{r},\vec{s})$; $\vec{b}:=\cdeg_{\vec{s}}\mathbf{M}$;
$k:=n-m;$}

\STATE{\textbf{$\left[\mathbf{M}_{1}^{T},\mathbf{M}_{2}^{T},\cdots,\mathbf{M}_{\left\lceil \log k\right\rceil -1}^{T},\mathbf{M}_{\left\lceil \log k\right\rceil }^{T}\right]:=\mathbf{M}$},
with $\mathbf{M}_{\left\lceil \log k\right\rceil },\mathbf{M}_{\left\lceil \log k\right\rceil -1},\cdots,\mathbf{M}_{2},\mathbf{M}_{1}$
having $\vec{s}$-row degrees in the range $\left[0,2\xi/k\right],(2\xi/k,4\xi/k],...,(\xi/4,\xi/2],(\xi/2,\xi].$\textbf{ }}

\FOR{$i$ \textbf{from $1$ to $\left\lceil \log k\right\rceil $ }} 

\forbody{\STATE{$\vec{\sigma}_{i}:=\left[\xi/2^{i-1}+1,\dots,\xi/2^{i-1}+1\right]$,
with the number of entries matches the row dimension of $\mathbf{M}_{i};$}}

\STATE{$\vec{\sigma}:=\left[\vec{\sigma}_{\left\lceil \log k\right\rceil },\vec{\sigma}_{\left\lceil \log k\right\rceil -1},\dots,\vec{\sigma}_{1}\right]$;}

\STATE{$\hat{\mathbf{M}}:=x^{\vec{\sigma}-\vec{b}-1}\mathbf{M};$}

\STATE{$\mathbf{N}_{0}:=I_{n}$; $\tilde{\mathbf{N}}_{0}:=I_{n};$}

\FOR{$i$ \textbf{from $1$ to $\left\lceil \log k\right\rceil $ }} 

\forbody{\STATE{$\vec{s}_{i}:=-\cdeg_{-\vec{s}_{i-1}}\mathbf{N}_{i-1};$ (note $\vec{s}_{1}=\vec{s}$)}

\STATE{$\mathbf{P}_{i}:=\umab\left(\hat{\mathbf{M}}_{i}\tilde{\mathbf{N}}_{i-1},\vec{\sigma}_{i},-\vec{s}_{i}\right)$; }

\STATE{$\left[\mathbf{N}_{i},\mathbf{Q}_{i}\right]:=\mathbf{P}_{i}$, where
$\mathbf{N}_{i}$ is a $\left(\hat{\mathbf{M}}_{i},-\vec{s}_{i}\right)$-kernel
basis;}

\STATE{$\tilde{\mathbf{N}_{i}}:=\tilde{\mathbf{N}}_{i-1}\cdot\mathbf{N}_{i};$ }

\STATE{$\mathbf{R}:=\left[\mathbf{R},\tilde{\mathbf{N}}_{i-1}\mathbf{Q}_{i}\right]$;}}

\STATE{$\mathbf{R}^{r}:=\colRev\left(\mathbf{R},-\vec{s},\cdeg_{-\vec{s}}\mathbf{R}\right);$}

\RETURN $\left(\mathbf{R}^{r}\right)^{T}$ 
\end{algorithmic}
\end{algorithm}



\subsection{Computational Cost}

The cost of \prettyref{alg:unimodularCompletion} is dominated by
the kernel basis computation, order basis computations, and the multiplications
$\hat{\mathbf{M}}_{i}\tilde{\mathbf{N}}_{i-1}$, $\tilde{\mathbf{N}}_{i-1}\mathbf{N}_{i}$,
and $\tilde{\mathbf{N}}_{i-1}\mathbf{Q}_{i}$. From \prettyref{thm:kerncomp}
we know the kernel basis computation can be done with a cost of $O^{\sim}\left(n^{\omega}s\right)$.
To determine the cost of order basis computations and multiplications,
it is helpful to first look at the size of $\vec{s}_{i}$. 
\begin{lem}
\label{lem:sSize}The shifted degrees $\vec{s}_{i}=-\cdeg_{-\vec{s}}\tilde{\mathbf{N}}_{i-1}=-\cdeg_{-\vec{s}_{i-1}}\mathbf{N}_{i-1}$satisfy
$\sum\vec{s}_{i}\le\xi$.\end{lem}
\begin{proof}
Recall that $\tilde{\mathbf{N}}_{i-1}$ is a $-\vec{s}$-minimal kernel
basis of a matrix $\mathbf{A}$ consists of a subset of rows of $\mathbf{M}^{T}$,
which has $\cdeg_{\vec{s}}\mathbf{M}=\vec{b}$, or $\rdeg_{\vec{s}}\mathbf{M}^{T}=\vec{b}$.
Hence by \prettyref{lem:generaKernelBasisDegreeBound} 
\[
\sum-\vec{s}_{i}=\cdeg_{-\vec{s}}\tilde{\mathbf{N}}_{i-1}\le\sum\rdeg_{\vec{s}}\mathbf{A}-\sum\vec{s}\le\sum\vec{b}-\sum\vec{s},
\]
which gives 
\[
\sum\vec{s}_{i}\le\sum\vec{s}-\sum\vec{b}\le\sum\vec{s}=\xi.
\]
Here recall that $\vec{b}=\cdeg_{\vec{s}}\mathbf{M}\ge0$ since $\vec{s}\ge0$
and $\vec{b}\le\sum\vec{s}$ by \prettyref{thm:boundOfSumOfShiftedDegreesOfKernelBasis}. 
\end{proof}
The order basis computation $\umab$ in \prettyref{alg:unimodularCompletion}
uses the algorithm for computing order basis with unbalanced shift
from \citep{zhou:phd2012,ZL2012}. 
\begin{lem}
An $\left(\hat{\mathbf{M}}_{i}\tilde{\mathbf{N}}_{i-1},\vec{\sigma}_{i},-\vec{s}_{i}\right)$-order
basis can be computed with a cost of $O^{\sim}\left(n^{\omega}s\right)$. \end{lem}
\begin{proof}
From the construction of $\hat{\mathbf{M}}_{i}$, the matrix $\hat{\mathbf{M}}_{i}$
and $\hat{\mathbf{M}}_{i}\tilde{\mathbf{N}}_{i-1}$ have less than
$2^{i}$ rows, and for simplicity can be assumed to be $2^{i}$ rows
by appending zero rows. We also have $\sigma_{i}=\left\lceil \xi/2^{i-1}\right\rceil +1\in\Theta\left(\xi/2^{i}\right)$.
From \prettyref{lem:sSize} we also have $\sum\vec{s}_{i}\le\xi$.
Therefore, the conditions of \prettyref{thm:unbalancedOrderBasisCost}
are satisfied, and Algorithm 2 from \citep{ZL2012} for order basis
computation with unbalanced shift can be used with a cost of $O^{\sim}\left(n^{\omega}s\right)$. \end{proof}
\begin{lem}
The multiplications $\hat{\mathbf{M}}_{i}\tilde{\mathbf{N}}_{i-1}$
can be done with a cost of $O^{\sim}\left(n^{\omega}s\right)$.\end{lem}
\begin{proof}
The dimension of $\hat{\mathbf{M}}_{i}$ is bounded by $2^{i}\times n$
and 
\[
\sum\rdeg_{\vec{s}}\hat{\mathbf{M}}_{i}\le2^{i}\cdot\xi/2^{i-1}\in O\left(\xi\right).
\]
As in the proof of \prettyref{lem:sSize} we also have $\cdeg_{-\vec{s}}\tilde{\mathbf{N}}_{i-1}\le0$,
or equivalently, $\rdeg\tilde{\mathbf{N}}_{i-1}\le\vec{s}$. We can
now use \prettyref{thm:multiplyUnbalancedMatrices} to multiply $\tilde{\mathbf{N}}_{i-1}^{T}$
and $\hat{\mathbf{M}}_{i}^{T}$ with a cost of $O^{\sim}\left(n^{\omega}s\right)$.\end{proof}
\begin{lem}
The multiplication $\tilde{\mathbf{N}}_{i-1}\mathbf{N}_{i}$ can be
done with a cost of $O^{\sim}\left(n^{\omega}s\right)$.\end{lem}
\begin{proof}
We know $\cdeg_{-\vec{s}}\tilde{\mathbf{N}}_{i-1}=-\vec{s}_{i}$,
and $\cdeg_{-\vec{s}_{i}}\mathbf{N}_{i}=-\vec{s}_{i+1}\le0.$ In other
words, $\rdeg\mathbf{N}_{i}\le\vec{s}_{i}$, and $\rdeg_{\vec{s}_{i}}\tilde{\mathbf{N}}_{i-1}\le\vec{s}$.
We also have $\sum\vec{s}_{i}\le\xi$ from \prettyref{lem:sSize}.
Hence we can again use \prettyref{thm:multiplyUnbalancedMatrices}
to multiply $\mathbf{N}_{i}^{T}$ and $\tilde{\mathbf{N}}_{i-1}^{T}$
with a cost of $O^{\sim}\left(n^{\omega}s\right)$.\end{proof}
\begin{lem}
The multiplication $\tilde{\mathbf{N}}_{i-1}\mathbf{Q}_{i}$ can be
done with a cost of $O^{\sim}\left(n^{\omega}s\right)$.\end{lem}
\begin{proof}
We know 
\[
\cdeg_{-\vec{s}_{i}}\mathbf{Q}_{i}\le\max\cdeg_{\vec{s}}\mathbf{P}=1,
\]
or equivalently, 
\[
\rdeg\mathbf{Q}_{i}\le\vec{s}_{i}+1.
\]
But we also know that this $\mathbf{Q}_{i}$ from the order basis
computation has a factor $xI$. Therefore, $\rdeg\left(\mathbf{Q}_{i}/x\right)\le\vec{s}_{i}$.
In addition, $\rdeg_{\vec{s}_{i}}\tilde{\mathbf{N}}_{i-1}\le\vec{s}$
as before. We again have $\sum\vec{s}_{i}\le\xi$ from \prettyref{lem:sSize}.
So we can again use \prettyref{thm:multiplyUnbalancedMatrices} to
multiply $\mathbf{Q}_{i}^{T}$ and $\tilde{\mathbf{N}}_{i-1}^{T}$
with a cost of $O^{\sim}\left(n^{\omega}s\right)$.\end{proof}
\begin{thm}
A unimodular completion of $\mathbf{F}$ can be computed with a cost
of $O^{\sim}\left(n^{\omega}s\right)$ field operations. 
\end{thm}

\section{Conclusion\label{sec:Future-Research}}

In this paper, we have presented an efficient deterministic algorithm
for a unimodular completion of a matrix of polynomnials. Our algorithm
computes a unimodular completion of an input matrix $\mathbf{F}\in\mathbb{K}\left[x\right]^{m\times n}$,
$m<n$ with a cost of $O^{\sim}\left(n^{\omega}s\right)$, where $s$
is the average of the $m$ largest column degrees of the input matrix.
Future directions of interest include efficient deterministic unimodular
completion in domains such as matrices of multivariate polynomials
and matrices of differential or, more generally, of Ore operators.

%\bibliographystyle{plainnat}
 \bibliographystyle{plain}
\bibliography{paper}

\end{document}
